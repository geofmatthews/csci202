\documentclass{article}
\usepackage[margin=1in]{geometry}
\usepackage{enumitem}
\usepackage{hyperref}
\usepackage{fancyvrb}
\usepackage{multicol}
\usepackage{color}
\usepackage{longtable}

\setlength{\parindent}{0pt}
\setlength{\parskip}{1ex}


\newcommand{\myline}{\end{description}\hrulefill\begin{description}}
\newcommand{\be}{\mbox{}\begin{itemize}}
\newcommand{\ee}{\end{itemize}}
           
\begin{document}
\centerline{\Large\bf CSCI 202: Dynamic Web Pages}
\centerline{\large\bf Syllabus, Spring, 2017}


\begin{description}


\item[Instructor:] Geoffrey Matthews\\
{\em Email:} {\tt geoffrey dot matthews at wwu dot edu}\\
{\em Office hours:} MTWF 10:00-10:50, CF 469


\item[Lectures:] MWF 9:00-9:50, CF 227


\item[Catalog copy:] Principles and technologies required to produce
  and distribute Internet (World Wide Web) content, with a focus on
  site architecture and client-side dynamic pages; an introduction to
  server-side processing. 


\item[Goals:]  On completion of this course, students will demonstrate
  basic understanding of:
\begin{itemize}
\item design and development of web pages using HTML and CSS
\item the programming language Javascript
\item the jQuery library
\item the Document Object Model (DOM) for working with web pages using
  Javascript
\item forms and regular expressions
\item data represented in XML and JSON
\item creating webpages on the server with PHP
\item interacting dynamically with a server using AJAX
\end{itemize}


\item[Websites:]\mbox{}
\begin{itemize}
  \item For class materials:
    \url{https://github.com/geofmatthews/csci202} 
  \item For turning in homework and grading:
    \url{https://wwu.instructure.com/}  
\end{itemize}

\item[Text:]\mbox{}
  
  \begin{itemize}
  \item {\em HTML5 and CSS3 All-In-One for Dummies}, Andy Harris.

    \url{http://www.wiley.com/WileyCDA/WileyTitle/productCd-1118289382.html}
  \item Readings from the internet as assigned.
  \end{itemize}
  

\item[Software:]\mbox{}
  
  \begin{itemize}
  \item \url{https://www.google.com/chrome/browser/desktop/index.html}

    For client-side programming: HTML, CSS, Javascript
  \item\url{https://www.apachefriends.org/index.html}

    For server-side programming: PHP, Ajax
  \end{itemize}

\item[Labs:]\mbox{}

  \begin{tabular}{|l|l|l|}\hline
    CRN & Time & Room \\\hline
    20333 & T 10:00-11:50am & CF 024 \\\hline
    20334 & T 2:00-3:50pm & CF 026\\\hline
  \end{tabular}
  
No labs the first and last week of class.

Labs must be turned in on canvas on or before midnight of
  the due date.  Specifications must be followed exactly.  No late
  work will be accepted.  It is the student's responsibility to make
  sure the {\em correct} file is submitted by midnight of the due
  date.

\item[Project:]  There will be a final project, which will be a
  website using most of what you have learned during the quarter.

\item[Exams:]\mbox{}
  
  \begin{description}
  \item[Midterm:]  Friday, May 5.
  \item[Final exam:] Wednesday, June 7, 10:30am - 12:30pm.
    \begin{itemize}
      \item Note:  this class does {\em not} include Tuesday or
        Thursday for final exam scheduling purposes,
        even if your lab meets on one of those days.
    \end{itemize}
  \end{description}
  
  The  exams are closed book, with the exception that you may consult
  two pieces of paper during the exam.  You may write or print whatever
  you wish on each side of these pieces of paper.

\item[Emergencies:] If you have to miss an assignment deadline or an
  exam because of a medical or other emergency notify me as soon as
  possible and we will negotiate a substitute.  If you know in advance
  that you have to miss a deadline, let me know at least a week in
  advance.

\item[Grading:] Students will be graded on the labs, the final
  project, the midterm, and the final.  There will also be
  some in-class pop quizzes (which are
  unscheduled and which we will solve together in class).

Grades will be assigned based on scores as shown.  At the discretion
of the instructor, scores may be scaled.  Awarding $\pm$ is at the
discretion of the instructor.

\begin{tabular}{|c|c|c|c|c|c|}\hline
Labs & Project & Pop quizzes & Midterm & Final \\\hline
30\% & 20\% & 5\% & 20\% & 25\% \\\hline
\end{tabular}

\begin{tabular}{|c|c|c|c|c|c|}\hline
\% & 90-100 & 80-89 & 70-79 & 60-69 & 0-60\\\hline
Grade & A & B & C & D & F\\\hline
\end{tabular}


\item[Academic dishonesty:] Please read Appendix D of WWU's Catalog on
  Academic Dishonesty.  It is available online at
  \url{http://catalog.wwu.edu}.

  Unless specified otherwise, all work for this course is meant to
  be done {\bf individually.}  The work that you turn in for a grade
  must be completely your own, or you will be guilty of academic
  dishonesty and could receive an F for the course.

  However, it can be a valiable learning experience to discuss
  work with your fellow students, and this is encouraged.
  However, after working with a colleague, {\bf you may not keep any
    paper or electronic copies of anything you produced together!}
  You may only keep your memories.  In particular, this means that
  {\bf you may not ask for or give help while sitting in front of a
    computer where the assignment is open!}  Also, {\bf you may not
    use anything a colleague has emailed to you!}  Delete the email
  and do not save a copy.

  To help understand what I mean, remember the {\fbox{\bf Long Term
    Memory Rule}}.  You may discuss, sketch, write things down, use
  your computers, whatever, but after you are done working with your
  fellow students all files must be deleted, whiteboards erased, and
  all papers you created must be destroyed.  You should then watch a
  rerun of {\em the Simpson's}, play a game of ping-pong, take a walk,
  or something else for half an hour. After this you can go back to
  your assignment (alone) and use the knowledge you have now gained.

  It is very easy for us to detect copied assignments.  Please do not
  put us in a situation where we have to fail you for plagiarism.

\item[Schedule:] Topics may change slightly as the course progresses,
  but I hope to cover the following sections of the textbook.  If we
  finish early we will spend class time discussing the development of
  your project.
  
  \begin{tabular}{|l|l|l|}\hline
    Subject & Textbook & Dates \\\hline
HTML & Part I & March 29 -- April 5 \\\hline
 CSS & Parts II \& III & April 7 -- April 14 \\\hline
 Javascript & Part IV & April 17 -- April 28 \\\hline
 PHP & Part V & May 1 -- May 12 \\\hline
 AJAX & Part VII & May 15 -- May 26 \\\hline
\end{tabular}
\end{description}


\end{document}
