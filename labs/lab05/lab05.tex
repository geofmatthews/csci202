\documentclass{article}
\usepackage[margin=1in]{geometry}
\usepackage{graphicx,hyperref}
\title{CSCI 202, Spring 2017, Lab \# 5}
\author{Geoffrey Matthews}

\begin{document}
\maketitle

\begin{description}

\item[Due date:] Midnight, Monday, May 15.

\item[Goals:] Javascript loops and arrays.

\item[Zip your folder:]
  Put all your files (including the supporting files) in a single
  folder and  make a compressed (zip) archive of the folder.

\item[Files to turn in:]\mbox{}
  \begin{itemize}
  \item
    One  {HTML} file: {\tt lab04.html}.
    It will be modified from the supplied {\tt penguins.html} file.
  \item
    One  style file, {\tt lab04.css}.  It
    will be modified from the supplied {\tt penguins.css} file.
\item One  Javascript file, {\tt lab04.js}, entirely of your own
  creation. 
\item The {\tt images} folder of supporting images should be included
  in your turnin folder, so that the page can be used immediately by
  the grader.
  \end{itemize}

\item[Lab steps:]\mbox{}
  \begin{itemize}
  \item This lab will be a continuation of the previous lab.  The idea
    is to make a {\em lot} of penguins (and one yeti), but with much
    less work.
  \item Make the width of this version the full page, so we can have 
many more    penguins in a row.  Note you'll have to fix the styling
of the banner.
\item 


  \end{itemize}
  


\item[Acknowledgements:] Supporting files and ideas for this lab came
  from\\ \url{https://googlecreativelab.github.io/coder-projects/}


\end{description}

\end{document}
