\documentclass{article}
\usepackage[margin=1in]{geometry}
\title{CSCI 202, Spring 2017, Lab \# 1}
\author{Geoffrey Matthews}

\begin{document}
\maketitle

\begin{description}

\item[Due date:] Midnight, Monday, April 17.

\item[Goals:] Get familiar with the basics of HTML.

\item[Zip your folder:]

  Put all your files (including the image file) in a single folder and
  make a compressed (zip) archive of the folder.
  \begin{itemize}
  \item
  On Windows,
  right-click on the folder, select ``Send to'', and then
  ``Compressed (zipped) folder.''
  \item
  On a Mac, right-click the folder, select ``Compress items''.
\item
  On linux, enter ``{\tt zip foldername}'' on the command line.
  \item
  On each system, you will then see a newly created file with
  the {\tt .zip} extension.  This is what you want to turn in to
  canvas.
  \end{itemize}
  
\item[Files:]  You will create five {\tt .html} files:
  \begin{description}
  \item[index.html]
    The title and {\tt h1} header of this page will be {\bf Main}.
    
    This file will have an ordered list of links to
    the other four files.

  \item[page01.html]The title and {\tt h1} header of this page will be
    {\bf Search Engines}.

    This file will have an unordered list of links to
    three search engines, such as {\bf google}.  Search the web for
    search engines, and remember to put their complete URL in the
    link.  (You can copy and paste links from your browser into your
    file.) 

  \item[page02.html] The title and {\tt h1} header of this page will
    be {\bf Image Scaling}.

    Find a nice image, not too big.  You can take one from the
    internet, provided you have rights to it, or use one of your own
    photos.

    Create a definition list with the terms {\bf Small}, {\bf
      Medium}, and {\bf Large}.  The item under each term will be your
    image, inline, resized to half its original dimensions under small,
    original size under medium, and double its original dimensions
    under large.

    The {\em same image} will be placed inline under each
    item, but it will be resized by the {\tt width} and {\tt height}
    attributes of the {\tt img} tag.

    Be sure to provide {\tt alt} text for each image.
    
  \item[page03.html] The title and {\tt h1} header of this page will
    be {\bf User Information}.

    On this page create an input form asking for the user's name,
    phone number, date of birth, and shoe size.
    
  \item[page04.html] The title and {\tt h1} header of this page will
    be {\bf Schedule}.

    On this page use a {\tt table} to create a schedule of your
    classes this quarter.  The headers of the table will be {\bf
      Class}, {\bf Hour}, {\bf Days}, {\bf Room}, and {\bf
      Instructor}.  One row of your table will contain the entries
    {\tt CSCI 202}, {\tt 9:00am}, {\tt MWF}, {\tt CF227}, and {\tt
      Matthews} .

  \end{description}      

  At the bottom of each of the four numbered pages you will also have
  a {\bf next} link, which will take you to the next page in a cyclic
  order ($1\rightarrow 2 \rightarrow 3\rightarrow 4\rightarrow
  1\rightarrow$ {\em etc.}), and
  a {\bf top} link, which takes you back to the index page.

  All pages should pass the W3C validator with no errors.

\end{description}

\end{document}
