\documentclass{article}
\usepackage[margin=1in]{geometry}
\usepackage{graphicx,hyperref,fancyvrb}
\title{CSCI 202, Spring 2017, Lab \# 4}
\author{Geoffrey Matthews}

\begin{document}
\maketitle

\begin{description}

\item[Due date:] Midnight, Monday, May 8.

\item[Goals:] Getting started with Javascript.

\item[Zip your folder:]
  Put all your files (including the supporting files) in a single
  folder and  make a compressed (zip) archive of the folder.

\item[Files to turn in:]\mbox{}
  \begin{itemize}
  \item
    One  {HTML} file: {\tt lab04.html}.
    It will be modified from the supplied {\tt penguins.html} file.
  \item
    One  style file, {\tt lab04.css}.  It
    will be modified from the supplied {\tt penguins.css} file.
\item One  Javascript file, {\tt lab04.js}, entirely of your own
  creation. 
\item The {\tt images} folder of supporting images should be included
  in your turnin folder, so that the page can be used immediately by
  the grader.
  \end{itemize}

\item[Lab steps:]\mbox{}
  \begin{itemize}
  \item Download and unzip the {\tt penguins.zip} file on the archive.
    \item
  After unzipping the {\tt penguins.zip} file you will find it
  contains HTML, CSS, and some supporting images.  Play with the web
  page a bit to get the idea of what is supposed to happen.
\item
  Copy the {\tt penguins.html} file to your own {\tt lab04.html} file.
\item
  Copy the {\tt penguins.css} file to your own {\tt lab04.css} file.
  Modify your {\tt lab04.html} file so it loads your {\tt lab04.css}
  script. 
\item
All of the interactivity for the penguins webpage is provided by 
the pseudo classes {\tt hover} and {\tt active}.  You are to replace
this with Javascript.
\item Delete the selectors in {\tt lab04.css} that use {\tt hover} and
  {\tt active}.
\item Change all of the ``class'' attributes of the penguins (and
  yeti) to ``id''.  This will make it easy for your javascript to find
  them. 
\item For each of the {\tt penguin} divs (and the yeti) in your {\tt
  lab04.html} file, add {\tt onmouseover}, {\tt onmouseout} and {\tt
  onclick} events, and have each one of them call an appropriately
  named Javascript function, {\em e.g.} {\tt mouseOver1, mouseOut1,
    click1, mouseOver2}, {\em etc.}
\item Create a Javascript file {\tt lab04.js}, and make sure you load
  this script in your {\tt lab04.html} file.
\end{itemize}
\item[Javascript coding:]\mbox{}
\begin{itemize}

\item In the Javascript file, create the functions needed by your
    penguins to change images to the question mark when the mouse
    enters, back to the default when the mouse leaves, and to the
    penguin (or yeti) when clicked.

  \item {\em Unlike} the original web page, your penguin (or yeti)
    will remain after clicking.

  \item In order for your Javascript to remember when a particular
    penguin has been clicked, you will use boolean variables, {\tt
      p1Clicked}, {\tt p2Clicked}, {\em etc.}  All of these will be
    set to {\tt false} at the top of the script.  Each of the
    functions will check this variable first, and not do anything if
    it has been clicked.

  \item The {\em clicked} function will set its variable
    ({\tt
      p1Clicked}, {\tt p2Clicked}, {\em etc.} )
    to {\tt
      true}, in addition to changing the image to the penguin (or
      yeti). 

\item Remember to set a style element with javascript, you use the
  following pattern.  If the style is set like this:
  \begin{Verbatim}[frame=single]
#foo { background-image : url('mypicture.png') }
  \end{Verbatim}
  Then the equivalent Javascript code would look like this:
  \begin{Verbatim}[frame=single]
document.getElementById("foo").style.backgroundImage = "url('mypicture.png')";
  \end{Verbatim}
  Observe the use of single and double quotes in the url, and the fact
  that style names such as {\tt background-image} have to be converted to
  camel case.
    
  \item {\bf Big Hint:} to make this a LOT easier, get it working for ONE
    penguin first!  Once you've got it correct, copying and
    pasting, and changing the penguin number will make the other
    penguins (and yeti) a lot easier.
  \end{itemize}
  
\item[Optional (for advanced students):]
  You can make this assignment a bit shorter by using
  arrays and function parameters.  For example, instead of the eight
  functions {\tt mouseOver1}, {\tt mouseOver2}, {\em etc.}, you can
  write a single function {\tt mouseOver} that takes a single
  parameter, the number of the penguin.

  If you do this, you can also approach the images in two ways.  One
  is to construct the appropriate image name from the number, for
  example: \verb|"images/mound_" + num + ".png"|.  Another, more
  general way, is to use an array of image names:
  \begin{Verbatim}
    ["images/mound_1.png", "images/mound_2.png", "images/mound_3.png", ...]
  \end{Verbatim}
  If you do this, remember that arrays in Javascript are indexed from
  0. 
\item[Acknowledgements:] Supporting files and ideas for this lab came
  from\\ \url{https://googlecreativelab.github.io/coder-projects/}


\end{description}

\end{document}
