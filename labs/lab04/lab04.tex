\documentclass{article}
\usepackage[margin=1in]{geometry}
\usepackage{graphicx,hyperref}
\title{CSCI 202, Spring 2017, Lab \# 4}
\author{Geoffrey Matthews}

\begin{document}
\maketitle

\begin{description}

\item[Due date:] Midnight, Monday, May 8.

\item[Goals:] Getting started with Javascript.

\item[Zip your folder:]
  Put all your files (including the supporting files) in a single
  folder and  make a compressed (zip) archive of the folder.

\item[Files to turn in:]\mbox{}
  \begin{itemize}
  \item
    One  {HTML} file: {\tt lab04.html}.
    It will be modified from the supplied {\tt penguins.html} file.
  \item
    One  style file, {\tt lab04.css}.  It
    will be modified from the supplied {\tt penguins.css} file.
\item One  Javascript file, {\tt lab04.js}, entirely of your own
  creation. 
\item The {\tt images} folder of supporting images should be included
  in your turnin folder, so that the page can be used immediately by
  the grader.
  \end{itemize}

\item[Lab steps:]\mbox{}
  \begin{itemize}
  \item Download and unzip the {\tt penguins.zip} file on the archive.
    \item
  After unzipping the {\tt penguins.zip} file you will find it
  contains HTML, CSS, and some supporting images.  Play with the web
  site a bit to get the idea of what is supposed to happen.
\item
  Copy the {\tt penguins.html} file to your own {\tt lab04.html} file.
\item
  Copy the {\tt penguins.css} file to your own {\tt lab04.css} file.
  Modify your {\tt lab04.html} file so it loads your {\tt lab04.css}
  script. 
\item
All of the interactivity for the penguins webpage is provided by 
the pseudo classes {\tt hover} and {\tt active}.  You are to replace
this with Javascript.
\item Delete the selectors in {\tt lab04.css} that use {\tt hover} and
  {\tt active}.
\item For each of the {\tt penguin} divs in your {\tt lab04.html}
  file, add {\tt onmouseover}, {\tt onmouseout} and
  {\tt onclick} events, and have each
  one of them call an appropriately named Javascript function.
\item Create a Javascript file {\tt lab04.js}, and make sure you load
  this script in your {\tt lab04.html} file.
\item In this Javascript file, create the functions needed by your
    penguins to change images to the question mark when the mouse
    enters, back to the default when the mouse leaves, and to the
    penguin (or yeti) when clicked.
    \item {\em Unlike} the original web page, your penguin (or yeti)
      will remain after clicking.
  \end{itemize}
  


\item[Acknowledgements:] Supporting files and ideas for this lab came
  from\\ \url{https://googlecreativelab.github.io/coder-projects/}


\end{description}

\end{document}
