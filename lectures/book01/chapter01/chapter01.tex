\documentclass{beamer}
\usepackage{fancyvrb}
\usepackage{hyperref}
\usepackage{alltt}
\usepackage{graphicx}

\newtheorem{theo}{Theorem}[section]


\newcommand{\myfig}[1]{\centerline{\includegraphics[scale=0.25]{figures/#1.png}}}
\newcommand{\myfigt}[1]{\centerline{\includegraphics[scale=0.2]{figures/#1.png}}}

\newcommand{\trans}[5]{
\begin{tabular}{|c|c|c|c|c|}\hline
#1 & #2 & #3 & #4 & #5 \\\hline
\end{tabular}
}

\newcommand{\arr}{&\rightarrow&}
\newcommand{\darr}{&\Rightarrow&}
\newcommand{\ar}{\ensuremath{\rightarrow}}
\newcommand{\dar}{\ensuremath{\Rightarrow}}
\newcommand{\bee}{\begin{eqnarray*}}
\newcommand{\eee}{\end{eqnarray*}}
\newcommand{\emptystring}{\ensuremath{\epsilon}}

\newcommand{\bi}{\begin{itemize}}
\newcommand{\li}{\item}
\newcommand{\ei}{\end{itemize}}

\newcommand{\sect}[1]{
\section{#1}
\begin{frame}[fragile]\frametitle{#1}
}
\newcommand{\sectc}[1]{
\section{#1}
\begin{frame}[fragile]\frametitle{#1}
\begin{columns}
}
\newcommand{\nc}[1]{\column{#1\textwidth}}
\newcommand{\ec}{\end{columns}}

\mode<presentation>
{
%  \usetheme{Madrid}
  % or ...

%  \setbeamercovered{transparent}
  % or whatever (possibly just delete it)
}

\usepackage[english]{babel}

\usepackage[latin1]{inputenc}

\title[HTML5 and CSS3, Chapter 1]
{
HTML5 and CSS3, Chapter 1
}

\subtitle{} % (optional)

\author[Geoffrey Matthews]
{Geoffrey Matthews}
% - Use the \inst{?} command only if the authors have different
%   affiliation.

\institute[WWU/CS]
{
  Department of Computer Science\\
  Western Washington University
}
% - Use the \inst command only if there are several affiliations.
% - Keep it simple, no one is interested in your street address.

\date{\today}

% If you have a file called "university-logo-filename.xxx", where xxx
% is a graphic format that can be processed by latex or pdflatex,
% resp., then you can add a logo as follows:

%\pgfdeclareimage[height=0.5cm]{university-logo}{WWULogoProColor}
%\logo{\pgfuseimage{university-logo}}

% If you wish to uncover everything in a step-wise fashion, uncomment
% the following command: 

%\beamerdefaultoverlayspecification{<+->}

\begin{document}

\begin{frame}
  \titlepage
\end{frame}


\newcommand{\myref}[1]{\small\item\url{#1}}
\newcommand{\myreft}[1]{\footnotesize\item\url{#1}}

%\begin{frame}
%  \frametitle{Outline}
%  \tableofcontents
%  % You might wish to add the option [pausesections]
%\end{frame}

\sect{Links}

\begin{itemize}

\item \url{www.aharrisbooks.net/haio}

  Book companion website
\item \url{www.dummies.com/cheatsheet/html5css3aio}

  Book cheatsheet


\end{itemize}

\end{frame}

\sect{My first web page}

  \begin{Verbatim}
<!DOCTYPE HTML>
<html lang="en-US">
<head>
<meta charset="UTF-8">
<!-- myFirst.html -->
<title>My very first web page!</title>
</head>
<body>
<h1>This is my first web page!</h1>
<p>
This is the first web page I've ever made,
and I'm extremely proud of it.
It is so cool!
</p>
</body>
</html>
\end{Verbatim}

\end{frame}

\sect{Points about HTML}
\bi
\item It uses plain text.
\item It works on all computers.
\item It describes what documents {\em mean}.
\item It does not describe how documents {\em look}.
\item It's easy to write.
\item It's free.
\ei

\end{frame}

\sect{DOCTYPE}
\bi
\li \verb|<!DOCTYPE HTML>|

\li
This special tag is used to inform the browser that
the document type is HTML. This is how the browser knows you'll be
writing an HTML5 document. You will sometimes see other values for
the doctype, but HTML5 is the way to go these days.
\ei
\end{frame}

\sect{html}
\bi
\li \verb|<html lang = "en"></html>|

\li
The \verb|<html>| tag is the foundation of
the entire web page. The tag begins the page. Likewise, \verb|</html>| ends
the page. For example, the page begins with \verb|<html>| and ends with
\verb|</html>|. The \verb|<html></html>|
combination indicates that everything
in the page is defined as HTML code.

In HTML5, you're expected to tell
the browser which language the page will be written in. Because I write
in English, I'm specifying with the code "en".

\ei
\end{frame}

\sect{head}

\bi
\li \verb|<head></head>|

\li
These tags define a special part of the web page called
the head (or sometimes header).  This is where you put some great stuff
later, but it's not where the main document lives. For now, the only thing
you'll put in the header is the document's title and charset.
Later, you'll add styling
information and programming code.
\ei
\end{frame}

\sect{meta}
\bi
\li \verb|<meta charset="UTF-8">|

\li
The \verb|meta| tag is used to tell the browser
which character set to use. Most of the web uses UTF-8. 

\li
\url{https://en.wikipedia.org/wiki/UTF-8}

\ei
\end{frame}

\sect{Comments}

\bi
\li \verb|<!--   -->|
\li This tag indicates a {\em comment}.  It is ignored by the browser
but may help the developer understand what's going on.

\ei

\end{frame}

\sect{title}
\bi
\li \verb|<title></title>|

\li
This tag is used to determine the page's title. The
title usually contains ordinary text. Whatever you define as the title will
appear in some special ways. Many browsers put the title text in the
browser's title bar. Search engines often use the title to describe the page.

\li Throughout the text, the filename of the HTML code is the
title.   Typically, you'll use something more descriptive.

\ei
\end{frame}


\sect{body}
\bi
\li \verb|<body></body>|

\li The page's main content is contained within these
tags. Most of the HTML code the user sees is in the body
area. 
\ei
\end{frame}

\sect{Headings}
\bi
\li \verb|<h1></h1>|

\li \verb|h1| stands for {\em heading level one}. Any text contained
within this markup is treated as a prominent headline. By default,
most browsers add special formatting to anything defined as \verb|h1|, but
there's no guarantee. An \verb|h1| heading doesn't really specify any
particular font or formatting, just the meaning of the text as a level
one heading.

\li There are also headings at lower levels, \verb|h2|, \verb|h3|, {\em etc.}
\ei
\end{frame}

\sect{Paragraphs}
\bi
\li \verb|<p>  </p>|

\li \verb|p| is a paragraph tag.  Newlines and whitespace are largely
ignored by HTML, because HTML is {\em not} designed to represent
appearance.
\ei
\end{frame}

\sect{General notes about tags}
\bi
\li Tags are lowercase.
\li Tag pairs are containers, with a beginning and an end.
\li Some elements can be repeated:
\bi
\li \verb|h1|
\li \verb|p|
\ei
\li Some elements should have only one:
\bi
\li \verb|html|
\li \verb|title|
\li \verb|body|
\ei
\li Extra whitespace, including carriage returns, is ignored.
\ei
\end{frame}

\sect{Test your pages on more than one browser}

\bi
\li Different browsers will make the same page look different.
\li {\bf Chrome} is a good browser for developers, and free.
\ei
\end{frame}

\sect{You never have complete control over the appearance}
\bi
\li The user may:
\bi
\li  have different size screen.
\li  use different browser.
\li  use different operating system.
\li  have a slow internet and turn off graphics.
\li  be blind and use screen-reader technology.
\li  be using a phone or tablet.
\li ...
\ei
\li Your document can only indicate how the information fits together
and make suggestions about the visual design.

\ei
\end{frame}

\end{document}
