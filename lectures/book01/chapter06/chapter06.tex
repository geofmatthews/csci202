\documentclass{beamer}
\usepackage{fancyvrb}
\usepackage{hyperref}
\usepackage{alltt}
\usepackage{graphicx}

\newtheorem{theo}{Theorem}[section]


\newcommand{\myfig}[1]{\centerline{\includegraphics[scale=0.25]{figures/#1.png}}}
\newcommand{\myfigt}[1]{\centerline{\includegraphics[scale=0.2]{figures/#1.png}}}

\newcommand{\trans}[5]{
\begin{tabular}{|c|c|c|c|c|}\hline
#1 & #2 & #3 & #4 & #5 \\\hline
\end{tabular}
}

\newcommand{\arr}{&\rightarrow&}
\newcommand{\darr}{&\Rightarrow&}
\newcommand{\ar}{\ensuremath{\rightarrow}}
\newcommand{\dar}{\ensuremath{\Rightarrow}}
\newcommand{\bee}{\begin{eqnarray*}}
\newcommand{\eee}{\end{eqnarray*}}
\newcommand{\emptystring}{\ensuremath{\epsilon}}

\newcommand{\bi}{\begin{itemize}}
\newcommand{\li}{\item}
\newcommand{\ei}{\end{itemize}}

\newcommand{\sect}[1]{
\section{#1}
\begin{frame}[fragile]\frametitle{#1}
}
\newcommand{\sectc}[1]{
\section{#1}
\begin{frame}[fragile]\frametitle{#1}
\begin{columns}
}
\newcommand{\nc}[1]{\column{#1\textwidth}}
\newcommand{\ec}{\end{columns}}

\mode<presentation>
{
%  \usetheme{Madrid}
  % or ...

%  \setbeamercovered{transparent}
  % or whatever (possibly just delete it)
}

\usepackage[english]{babel}

\usepackage[latin1]{inputenc}

\title[HTML5 and CSS3, Chapter 6]
{
HTML5 and CSS3, Chapter 6
}

\subtitle{} % (optional)

\author[Geoffrey Matthews]
{Geoffrey Matthews}
% - Use the \inst{?} command only if the authors have different
%   affiliation.

\institute[WWU/CS]
{
  Department of Computer Science\\
  Western Washington University
}
% - Use the \inst command only if there are several affiliations.
% - Keep it simple, no one is interested in your street address.

\date{\today}

% If you have a file called "university-logo-filename.xxx", where xxx
% is a graphic format that can be processed by latex or pdflatex,
% resp., then you can add a logo as follows:

%\pgfdeclareimage[height=0.5cm]{university-logo}{WWULogoProColor}
%\logo{\pgfuseimage{university-logo}}

% If you wish to uncover everything in a step-wise fashion, uncomment
% the following command: 

%\beamerdefaultoverlayspecification{<+->}

\begin{document}

\begin{frame}
  \titlepage
\end{frame}


\newcommand{\myref}[1]{\small\item\url{#1}}
\newcommand{\myreft}[1]{\footnotesize\item\url{#1}}

%\begin{frame}
%  \frametitle{Outline}
%  \tableofcontents
%  % You might wish to add the option [pausesections]
%\end{frame}

\sect{Source code from book}

\bi
\li \url{http://www.aharrisbooks.net/haio/}
\ei
\end{frame}

\sect{Main uses of images}

\bi
\li External link
\li Embedded images
\li Background images
\li Custom bullets
\ei

\end{frame}

\sect{External links}
\begin{Verbatim}[frame=single]
<h1>Linking to an External Image</h1>
<p>
  <a href = "shipStandard.jpg">
  Susan B. Constant
  </a>
</p>
\end{Verbatim}
\bi
\li Are easy to create.
\li They don't preview the image.
\li They interrupt the flow.
\li The user must use the back button to return to the original page.
\ei
\end{frame}

\sect{Inline images}

\begin{Verbatim}[frame=single]
<h1>The Susan B. Constant</h1>
<p>
  <img src = "shipStandard.jpg"
       height = "480"
       width = "640"
       alt = "Susan B. Constant" />
</p>
\end{Verbatim}
\bi
\li \verb|<img ... />| tag has only attributes, there is no \verb|</img>|.
\li {\tt src} can be absolute or relative.
\li {\tt height} and {\tt width} should not be used to resize images
\bi\li  Browser can format the page before the image downloads\ei
\li {\tt alt} used by screen readers and search engines.
\bi\li {\tt alt} is {\em required} on all images\ei
\li {\tt <img />} is an inline tag, so it must be inside a block tag.
\ei

\end{frame}

\sect{Images}
\bi
\li My current camera's resolution: $3648 \times 2736$.
\li My computer monitor's resolution: $1680 \times 1050$.
\li Some browsers will just show you a corner of the image, with
scrollbars.     
\li Some browsers will automatically resize the image to fit,
and will show full image on click.
\li Downloading and resizing large images takes time.
\li Use image processing software to resize images for the web.
\bi
\li IrfanView, XnView, GimpShop
\li $320\times 240$
\li $640\times 480$
\li $1024\times 768$
\ei
\li Use a compressed format, such as {\tt jpg}
\li Learn how to do batch processing.
\ei
\end{frame}

\sect{Image formats}
\begin{description}
  \li[gif] Used for transparency or animation.  Don't use for photos.
  \li[jpg] Good for photos.  Text in {\tt jpg} can become blurry.
  \li[png] Good for most situations.
  \li[svg] Vector graphics, not images.  Can be re-sized without loss
  of quality.
  \li[bmp] Avoid this and other legacy formats.
\end{description}
\end{frame}

\sect{Audio}
\begin{Verbatim}[frame=single]
<audio controls = "controls">
  <source src = "Allemande.mp3" type = "audio/mpeg">
  <source src = "Allemande.ogg" type = "audio/ogg">
    Your browser does not support HTML5 Audio
    Please use this link instead:
  <a href = "Allemande.mp3">Allemande.mp3</a>
</audio>
\end{Verbatim}
\bi
\li Different browsers support different formats.
\li Provide alternatives for the {\tt source}.
\li Anything after the sources will be used if the browser can't
handle any of the available formats.
\li Audio processing software:  \url{http://www.audacityteam.org/}

\ei
\end{frame}


\sect{Video}
\begin{Verbatim}[frame=single]
<video src = "bigBuck.ogv"
       controls = "controls">
  Your browser does not support embedded video
  through HTML 5.
</video>
\end{Verbatim}
\bi
\li Videos are extremely large.
\li A better alternative is a link to YouTube.
\ei

\end{frame}

\sect{Intellectual property}
\bi
\li Be sure you have permission to use images, audio, and video from
the web.
\li Just because you {\em can} embed some artwork, doesn't mean you
    {\em should}.
\ei
\end{frame}    

\end{document}
