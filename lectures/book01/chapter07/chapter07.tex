\documentclass{beamer}
\usepackage{fancyvrb}
\usepackage{hyperref}
\usepackage{alltt}
\usepackage{graphicx}

\newtheorem{theo}{Theorem}[section]


\newcommand{\myfig}[1]{\centerline{\includegraphics[scale=0.25]{figures/#1.png}}}
\newcommand{\myfigt}[1]{\centerline{\includegraphics[scale=0.2]{figures/#1.png}}}

\newcommand{\trans}[5]{
\begin{tabular}{|c|c|c|c|c|}\hline
#1 & #2 & #3 & #4 & #5 \\\hline
\end{tabular}
}

\newcommand{\arr}{&\rightarrow&}
\newcommand{\darr}{&\Rightarrow&}
\newcommand{\ar}{\ensuremath{\rightarrow}}
\newcommand{\dar}{\ensuremath{\Rightarrow}}
\newcommand{\bee}{\begin{eqnarray*}}
\newcommand{\eee}{\end{eqnarray*}}
\newcommand{\emptystring}{\ensuremath{\epsilon}}

\newcommand{\bi}{\begin{itemize}}
\newcommand{\li}{\item}
\newcommand{\ei}{\end{itemize}}

\newcommand{\sect}[1]{
\section{#1}
\begin{frame}[fragile]\frametitle{#1}
}
\newcommand{\sectc}[1]{
\section{#1}
\begin{frame}[fragile]\frametitle{#1}
\begin{columns}
}
\newcommand{\nc}[1]{\column{#1\textwidth}}
\newcommand{\ec}{\end{columns}}

\mode<presentation>
{
%  \usetheme{Madrid}
  % or ...

%  \setbeamercovered{transparent}
  % or whatever (possibly just delete it)
}

\usepackage[english]{babel}

\usepackage[latin1]{inputenc}

\title[HTML5 and CSS3, Chapter 7]
{
HTML5 and CSS3, Chapter 7
}

\subtitle{} % (optional)

\author[Geoffrey Matthews]
{Geoffrey Matthews}
% - Use the \inst{?} command only if the authors have different
%   affiliation.

\institute[WWU/CS]
{
  Department of Computer Science\\
  Western Washington University
}
% - Use the \inst command only if there are several affiliations.
% - Keep it simple, no one is interested in your street address.

\date{\today}

% If you have a file called "university-logo-filename.xxx", where xxx
% is a graphic format that can be processed by latex or pdflatex,
% resp., then you can add a logo as follows:

%\pgfdeclareimage[height=0.5cm]{university-logo}{WWULogoProColor}
%\logo{\pgfuseimage{university-logo}}

% If you wish to uncover everything in a step-wise fashion, uncomment
% the following command: 

%\beamerdefaultoverlayspecification{<+->}

\begin{document}

\begin{frame}
  \titlepage
\end{frame}


\newcommand{\myref}[1]{\small\item\url{#1}}
\newcommand{\myreft}[1]{\footnotesize\item\url{#1}}

%\begin{frame}
%  \frametitle{Outline}
%  \tableofcontents
%  % You might wish to add the option [pausesections]
%\end{frame}

\sect{Source code from book}

\bi
\li \url{http://www.aharrisbooks.net/haio/}
\ei
\end{frame}

\sect{Forms}
\bi
\li Forms are used to provide input from the user.
\li This section of the book shows how to build forms.
\li In order to {\em do} something with the data, we need a
programming language.
\li In Part IV we use Javascript on the client to do something.
\li in Part V we use PHP on the server to do something.
\li Consider this part a reference for later.
\ei
\end{frame}

\sect{Types of forms}
\bi
\li Text boxes
\li Password boxes
\li Text areas
\li Select lists
\li Check boxes
\li Radio buttons
\li Buttons
\li Labels
\li Fieldsets and legends
\ei
\end{frame}

\sect{Basic elements of a form}
\begin{Verbatim}[frame=single]
    <h1>A basic form</h1>
    <form action = "">
      <h2>Form elements go here</h2>
      <h3>Other HTML is fine, too.</h3>
      <p>
        <input type = "text"
             value = "googoo" />
      </p>
    </form>
    
    <p>
      Note this code is slightly improved from book
      version. I've placed the input
      element inside the form
      tag.  Thanks to Jim for the catch!
    </p>
\end{Verbatim}  

\end{frame}    

\sect{Organizing a form with fieldsets}
\begin{Verbatim}[frame=single]
    <h1>Sample Form with a Fieldset</h1>
    <form action = "">
      <fieldset>
        <legend>Personal Data</legend>
        <p>
          <label>Name</label>
          <input type = "text" />
        </p>
        <p>
          <label>Address</label>
          <input type = "text" />
        </p>
        <p>
          <label>Phone</label>
          <input type = "text" />
        </p>
      </fieldset>
    </form>
\end{Verbatim}
\end{frame}

\sect{Fieldsets}
\bi
\li Not necessary, but group parts of a form.
\li Legend identifies the entire form.
\li Label identifies each input form.
\li Can use paragraphs to group labels and inputs.
\li Label tag can have a {\tt for} attribute to connect with a
specific input.
\ei
\end{frame}

\sect{A standard text form}
\begin{Verbatim}[frame=single]
    <form action = "">
      <p>
        <label>Name</label>
        <input type = "text"
               id = "txtName"
               value = "Jonas"/>
               </p>
    </form>
\end{Verbatim}
\end{frame}

\sect{Text form attributes}
\bi
\li {\tt type}: determines the general type of input.
\bi\li text\li password\li {\em etc.}\ei
\li {\tt id}: when we deal with the data, the program has to know the
name (identifier) of the data.
\li {\tt value:} default value.
\li {\tt size:} number of characters displayed.
\li {\tt maxlength:} largest number of characters allowed.
\ei
\end{frame}

\sect{Password}
\begin{Verbatim}[frame=single]
      <fieldset>
        <legend>Enter a password</legend>
        <p>
          <label>Type password here</label>
          <input type = "password"
                 id = "pwd"
                 value = "secret" />
        </p>
      </fieldset>
\end{Verbatim}
\bi
\li Replaces visible text characters with asterisks.
\li Provices {\em no} security except for people looking over your
shoulder.
\li SSL (Secure Socket Layer) provides real security.
\ei

\end{frame}


\sect{Multi-line text input}
\begin{Verbatim}[frame=single]
        <p>
          <label>
            Please enter the sum total of
            Western thought. Be brief.
          </label>
        </p>
        <p>
          <textarea id = "txtAnswer"
                    rows = "10"
                    cols = "40"></textarea>
        </p>
\end{Verbatim}

\bi
\li It needs an {\tt id}
\li Specify the size with {\tt rows} and {\tt cols}.
\li The content goes between the tags.
\li Whitespace between the tags goes in the output.
\ei
\end{frame}

\sect{Drop-down selections}
\begin{Verbatim}[frame=single]
<label>What is your favorite color?</label>
<select id = "selColor">
  <option value = "#ff0000">Red</option>
  <option value = "#00ff00">Green</option>
  <option value = "#0000ff">Blue</option>
  <option value = "#00ffff">Cyan</option>
  <option value = "#ff00ff">Magenta</option>
  <option value = "#ffff00">Yellow</option>
  <option value = "#000000">Black</option>
  <option value = "#ffffff">White</option>
</select>
\end{Verbatim}
\bi
\li Saves screen space.
\li Limits input.
\li The value can be different from what the user sees.
\ei
\end{frame}


\sect{Check boxes}
\begin{Verbatim}[frame=single]
        <p>
          <input type = "checkbox"
                 id = "chkPeace"
                 value = "peace" />
          World peace
        </p>
        <p>
          <input type = "checkbox"
                 id = "chkHarmony"
                 value = "harmony" />
          Harmony and brotherhood
        </p>
        <p>
          <input type = "checkbox"
                 id = "chkCash"
                 value = "cash" />
          Cash
        </p>
\end{Verbatim}
\end{frame}

\sect{This all seems inconsistent}
\bi
\li Forms are inconsistent:
\bi 
\li Sometimes the value of a form element is visible to the users,
 sometimes not.
\li Sometimes the text the user sees is inside the tag,
sometimes not.
\ei
\li The standards for HTML evolved over time.
\li It's difficult to change a pattern that has thousands or millions
of uses.
\li It's best to consult an example any time you have to use
one of thses.
\ei
\end{frame}

\sect{Radio buttons}
\begin{Verbatim}[frame=single]
   <p>    <input type = "radio"
                 name = "radPrice"
                 id = "rad100"
                 value = "100" />Too much       </p>
   <p>    <input type = "radio"
                 name = "radPrice"
                 id = "rad200"
                 value = "200" />Way too much   </p>
   <p>    <input type = "radio"
                 name = "radPrice"
                 id = "rad5000"
                 value = "5000"
                 checked = "checked" />
          You've got to be kidding.             </p>
        \end{Verbatim}
\end{frame}

\sect{Radio buttons}
\bi
\li Only one can be checked at a time.
\li They have to be in a group.
\li They have to be the same name!
\bi\li This creates the group. \ei
\li You can have more than one group, just use different names.
\li One of them has to be checked.
\ei
\end{frame}

\sect{Other buttons}
\begin{Verbatim}[frame=single]
        <legend>
          input-style buttons
        </legend>
        <input type = "button"
               value = "input type = button" />
        <input type = "submit" />
        <input type = "reset" />
      </fieldset>
      <fieldset>
        <legend>button tag buttons</legend>
        <button type = "button">
          button tag
        </button>
        <button>
          <img src = "clickMe.gif"
               alt = "click me" />
        </button>
\end{Verbatim}
\end{frame}

\sect{Button tags {\em vs.} button attributes}
\bi
\li The {\tt type} determines the style.
\li The caption goes in the block.
\li You can incorporate other elements.
\ei
\end{frame}

\sect{New form input types}
\bi
\li date
\li time
\li datetime
\li datetime-local
\li week
\li month
\li color
\li number
\li range
\li search
\li email
\li tel
\li url
\ei
\end{frame}



\end{document}

