\documentclass{beamer}
\usepackage{fancyvrb}
\usepackage{hyperref}
\usepackage{alltt}
\usepackage{graphicx}

\newtheorem{theo}{Theorem}[section]


\newcommand{\myfig}[1]{\centerline{\includegraphics[scale=0.25]{figures/#1.png}}}
\newcommand{\myfigt}[1]{\centerline{\includegraphics[scale=0.2]{figures/#1.png}}}

\newcommand{\trans}[5]{
\begin{tabular}{|c|c|c|c|c|}\hline
#1 & #2 & #3 & #4 & #5 \\\hline
\end{tabular}
}

\newcommand{\arr}{&\rightarrow&}
\newcommand{\darr}{&\Rightarrow&}
\newcommand{\ar}{\ensuremath{\rightarrow}}
\newcommand{\dar}{\ensuremath{\Rightarrow}}
\newcommand{\bee}{\begin{eqnarray*}}
\newcommand{\eee}{\end{eqnarray*}}
\newcommand{\emptystring}{\ensuremath{\epsilon}}

\newcommand{\bi}{\begin{itemize}}
\newcommand{\li}{\item}
\newcommand{\ei}{\end{itemize}}

\newcommand{\sect}[1]{
\section{#1}
\begin{frame}[fragile]\frametitle{#1}
}
\newcommand{\sectc}[1]{
\section{#1}
\begin{frame}[fragile]\frametitle{#1}
\begin{columns}
}
\newcommand{\nc}[1]{\column{#1\textwidth}}
\newcommand{\ec}{\end{columns}}

\mode<presentation>
{
%  \usetheme{Madrid}
  % or ...

%  \setbeamercovered{transparent}
  % or whatever (possibly just delete it)
}

\usepackage[english]{babel}

\usepackage[latin1]{inputenc}

\title[HTML5 and CSS3, Chapter 7]
{
HTML5 and CSS3, Chapter 7
}

\subtitle{} % (optional)

\author[Geoffrey Matthews]
{Geoffrey Matthews}
% - Use the \inst{?} command only if the authors have different
%   affiliation.

\institute[WWU/CS]
{
  Department of Computer Science\\
  Western Washington University
}
% - Use the \inst command only if there are several affiliations.
% - Keep it simple, no one is interested in your street address.

\date{\today}

% If you have a file called "university-logo-filename.xxx", where xxx
% is a graphic format that can be processed by latex or pdflatex,
% resp., then you can add a logo as follows:

%\pgfdeclareimage[height=0.5cm]{university-logo}{WWULogoProColor}
%\logo{\pgfuseimage{university-logo}}

% If you wish to uncover everything in a step-wise fashion, uncomment
% the following command: 

%\beamerdefaultoverlayspecification{<+->}

\begin{document}

\begin{frame}
  \titlepage
\end{frame}


\newcommand{\myref}[1]{\small\item\url{#1}}
\newcommand{\myreft}[1]{\footnotesize\item\url{#1}}

%\begin{frame}
%  \frametitle{Outline}
%  \tableofcontents
%  % You might wish to add the option [pausesections]
%\end{frame}

\sect{Source code from book}

\bi
\li \url{http://www.aharrisbooks.net/haio/}
\ei
\end{frame}

\sect{Forms}
\bi
\li Forms are used to provide input from the user.
\li This section of the book shows how to build forms.
\li In order to {\em do} something with the data, we need a
programming language.
\li In Part IV we use Javascript on the client to do something.
\li in Part V we use PHP on the server to do something.
\li Consider this part a reference for later.
\ei
\end{frame}

\sect{Types of forms}
\bi
\li Text boxes
\li Password boxes
\li Text areas
\li Select lists
\li Check boxes
\li Radio buttons
\li Buttons
\li Labels
\li Fieldsets and legends
\ei
\end{frame}

\sect{Basic elements of a form}
\begin{Verbatim}[frame=single]
    <h1>A basic form</h1>
    <form action = "">
      <h2>Form elements go here</h2>
      <h3>Other HTML is fine, too.</h3>
      <p>
        <input type = "text"
             value = "googoo" />
      </p>
    </form>
    
    <p>
      Note this code is slightly improved from book
      version. I've placed the input
      element inside the form
      tag.  Thanks to Jim for the catch!
    </p>
\end{Verbatim}  

\end{frame}    

\sect{Organizing a form with fieldsets}
\begin{Verbatim}[frame=single]
    <h1>Sample Form with a Fieldset</h1>
    <form action = "">
      <fieldset>
        <legend>Personal Data</legend>
        <p>
          <label>Name</label>
          <input type = "text" />
        </p>
        <p>
          <label>Address</label>
          <input type = "text" />
        </p>
        <p>
          <label>Phone</label>
          <input type = "text" />
        </p>
      </fieldset>
    </form>
\end{Verbatim}
\end{frame}

\sect{Fieldsets}
\bi
\li Not necessary, but group parts of a form.
\li Legend identifies the entire form.
\li Label identifies each input form.
\li Can use paragraphs to group labels and inputs.
\li Label tag can have a {\tt for} attribute to connect with a
specific input.
\ei
\end{frame}


\end{document}

