\documentclass{beamer}
\usepackage{fancyvrb}
\usepackage{hyperref}
\usepackage{alltt}
\usepackage{graphicx}

\newtheorem{theo}{Theorem}[section]


\newcommand{\myfig}[1]{\centerline{\includegraphics[scale=0.25]{figures/#1.png}}}
\newcommand{\myfigt}[1]{\centerline{\includegraphics[scale=0.2]{figures/#1.png}}}

\newcommand{\trans}[5]{
\begin{tabular}{|c|c|c|c|c|}\hline
#1 & #2 & #3 & #4 & #5 \\\hline
\end{tabular}
}

\newcommand{\arr}{&\rightarrow&}
\newcommand{\darr}{&\Rightarrow&}
\newcommand{\ar}{\ensuremath{\rightarrow}}
\newcommand{\dar}{\ensuremath{\Rightarrow}}
\newcommand{\bee}{\begin{eqnarray*}}
\newcommand{\eee}{\end{eqnarray*}}
\newcommand{\emptystring}{\ensuremath{\epsilon}}

\newcommand{\bi}{\begin{itemize}}
\newcommand{\li}{\item}
\newcommand{\ei}{\end{itemize}}

\newcommand{\sect}[1]{
\section{#1}
\begin{frame}[fragile]\frametitle{#1}
}
\newcommand{\sectc}[1]{
\section{#1}
\begin{frame}[fragile]\frametitle{#1}
\begin{columns}
}
\newcommand{\nc}[1]{\column{#1\textwidth}}
\newcommand{\ec}{\end{columns}}

\mode<presentation>
{
%  \usetheme{Madrid}
  % or ...

%  \setbeamercovered{transparent}
  % or whatever (possibly just delete it)
}

\usepackage[english]{babel}

\usepackage[latin1]{inputenc}

\title[HTML5 and CSS3, Chapter 5]
{
HTML5 and CSS3, Chapter 5
}

\subtitle{} % (optional)

\author[Geoffrey Matthews]
{Geoffrey Matthews}
% - Use the \inst{?} command only if the authors have different
%   affiliation.

\institute[WWU/CS]
{
  Department of Computer Science\\
  Western Washington University
}
% - Use the \inst command only if there are several affiliations.
% - Keep it simple, no one is interested in your street address.

\date{\today}

% If you have a file called "university-logo-filename.xxx", where xxx
% is a graphic format that can be processed by latex or pdflatex,
% resp., then you can add a logo as follows:

%\pgfdeclareimage[height=0.5cm]{university-logo}{WWULogoProColor}
%\logo{\pgfuseimage{university-logo}}

% If you wish to uncover everything in a step-wise fashion, uncomment
% the following command: 

%\beamerdefaultoverlayspecification{<+->}

\begin{document}

\begin{frame}
  \titlepage
\end{frame}


\newcommand{\myref}[1]{\small\item\url{#1}}
\newcommand{\myreft}[1]{\footnotesize\item\url{#1}}

%\begin{frame}
%  \frametitle{Outline}
%  \tableofcontents
%  % You might wish to add the option [pausesections]
%\end{frame}

\sect{Source code from book}

\bi
\li \url{http://www.aharrisbooks.net/haio/}
\ei
\end{frame}

\sect{Hyperlinks}
\bi
\li Any page can link to any other page.
\li Clicking a hyperlink takes you to the other page.
\li Some text or image or other element must be the trigger.
\li The browser needs to know where to go.
\li Links should look like links.
  \bi
  \li Default is underlined blue text
  \li After they've been visited they change to purple.
  \ei
\li You can change the link's appearance, \\
but make sure the user can still recognize it as a link!
\ei

\end{frame}

\sect{The anchor tag}
\begin{Verbatim}
One of my favorite websites is called
<a href = "http://www.wikipedia.org">wikipedia.</a>
This is a terrific site.
\end{Verbatim}
\bi
\li The web address is an attribute, in quotes.
\li The highlighted text occurs between the \verb|<a>| and the \verb|</a>|.
\li The anchor tag is embedded into normal text.
\li The text will flow around the anchor.
\li This is an {\bf inline} element.
\li Other tags, such as \verb|<h1>| and \verb|<p>| are {\bf block-level}.
\ei

\end{frame}

\sect{URLs}

\bi
\li
\begin{Verbatim}[frame=single]
http://www.google.com
\end{Verbatim}
\li URL:  Uniform Resource Locator.
\li URI:  Uniform Resource Identifier.
\li Protocol: \verb|http://|
\li Host name: \verb|www|
\bi\li The name of the computer
\li Does not have to be {\tt www}\ei
\li Domain name: \verb|.com|
\li Subdomain: \verb|google|
\bi\li Everything between the {\tt www.} and the {\tt .com}\ei
\ei

\end{frame}

\sect{URLs with pages}
\bi
\li \begin{Verbatim}[frame=single]
https://cse.wwu.edu/computer-science/matthews
\end{Verbatim}
\li Sometimes the address specifies a particular document on the
server.
\li Document name can include path, including containing folders.
\li Page name is usually optional.  Many servers have a default:
\bi
\li {\tt index.html}
\li {\tt home.htm}
\ei
\ei
\end{frame}

\sect{Common domains}
\begin{center}
\begin{tabular}{ll}
  Domain & Explanation \\\hline
  {\tt .org} & Non-profit institution \\\hline
  {\tt .com} & Commercial enterprise \\\hline
  {\tt .edu} & Educational institution \\\hline
  {\tt .gov} & Governmental institution \\\hline
  {\tt .ca} & Canada \\\hline
  {\tt .uk} & United Kingdom \\\hline
  {\tt .tv} & Tuvalu
\end{tabular}
\end{center}

\end{frame}


\sect{Lists of links}
\begin{Verbatim}
<ul>
  <li><a href = "http://www.aharrisbooks.net/haio">
      HTML / CSS / JavaScript ALL in One for Dummies</a>
      A complete resource to web development</li>
  <li><a href = "http://www.aharrisbooks.net/jad">
      JavaScript / AJAX for Dummies</a>
      Using JavaScript, AJAX, and jQuery</li>
  <li><a href="http://www.aharrisbooks.net/pythonGame">
      Game Programming - the L Line</a>
      Game development in Python</li>
  <li><a href="http://www.aharrisbooks.net/h5g">
      HTML5 Game Development for Dummies</a>
      Building web and mobile games in HTML5</li>
</ul>
\end{Verbatim}
\bi
\li Indentation can keep things clear.
\ei

\end{frame}

\sect{Absolute {\em vs.} Relative References}

\begin{Verbatim}
  <a href="https://www.google.com/">
  Google, on another server.
  </a>


  <a href="anotherpage.html">
  A page, on the same server, same folder.
  </a>


  <a href="myfolder/page.html">
  A page, on this server, in a subfolder.
  </a>
\end{Verbatim}
\end{frame}




\end{document}
