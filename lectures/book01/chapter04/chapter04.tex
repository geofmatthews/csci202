\documentclass{beamer}
\usepackage{fancyvrb}
\usepackage{hyperref}
\usepackage{alltt}
\usepackage{graphicx}

\newtheorem{theo}{Theorem}[section]


\newcommand{\myfig}[1]{\centerline{\includegraphics[scale=0.25]{figures/#1.png}}}
\newcommand{\myfigt}[1]{\centerline{\includegraphics[scale=0.2]{figures/#1.png}}}

\newcommand{\trans}[5]{
\begin{tabular}{|c|c|c|c|c|}\hline
#1 & #2 & #3 & #4 & #5 \\\hline
\end{tabular}
}

\newcommand{\arr}{&\rightarrow&}
\newcommand{\darr}{&\Rightarrow&}
\newcommand{\ar}{\ensuremath{\rightarrow}}
\newcommand{\dar}{\ensuremath{\Rightarrow}}
\newcommand{\bee}{\begin{eqnarray*}}
\newcommand{\eee}{\end{eqnarray*}}
\newcommand{\emptystring}{\ensuremath{\epsilon}}

\newcommand{\bi}{\begin{itemize}}
\newcommand{\li}{\item}
\newcommand{\ei}{\end{itemize}}

\newcommand{\sect}[1]{
\section{#1}
\begin{frame}[fragile]\frametitle{#1}
}
\newcommand{\sectc}[1]{
\section{#1}
\begin{frame}[fragile]\frametitle{#1}
\begin{columns}
}
\newcommand{\nc}[1]{\column{#1\textwidth}}
\newcommand{\ec}{\end{columns}}

\mode<presentation>
{
%  \usetheme{Madrid}
  % or ...

%  \setbeamercovered{transparent}
  % or whatever (possibly just delete it)
}

\usepackage[english]{babel}

\usepackage[latin1]{inputenc}

\title[HTML5 and CSS3, Chapter 4]
{
HTML5 and CSS3, Chapter 4
}

\subtitle{} % (optional)

\author[Geoffrey Matthews]
{Geoffrey Matthews}
% - Use the \inst{?} command only if the authors have different
%   affiliation.

\institute[WWU/CS]
{
  Department of Computer Science\\
  Western Washington University
}
% - Use the \inst command only if there are several affiliations.
% - Keep it simple, no one is interested in your street address.

\date{\today}

% If you have a file called "university-logo-filename.xxx", where xxx
% is a graphic format that can be processed by latex or pdflatex,
% resp., then you can add a logo as follows:

%\pgfdeclareimage[height=0.5cm]{university-logo}{WWULogoProColor}
%\logo{\pgfuseimage{university-logo}}

% If you wish to uncover everything in a step-wise fashion, uncomment
% the following command: 

%\beamerdefaultoverlayspecification{<+->}

\begin{document}

\begin{frame}
  \titlepage
\end{frame}


\newcommand{\myref}[1]{\small\item\url{#1}}
\newcommand{\myreft}[1]{\footnotesize\item\url{#1}}

%\begin{frame}
%  \frametitle{Outline}
%  \tableofcontents
%  % You might wish to add the option [pausesections]
%\end{frame}

\sect{Source code from book}

\bi
\li \url{http://www.aharrisbooks.net/haio/}
\ei
\end{frame}

\sect{Unordered list}
\bi
\li Default behavior:
\bi
\li Items are indented
\li Each item has a bullet
\li Each item starts a new line
\ei
\li This is only one way of viewing an unordered list.
\li Appearance can be changed with CSS.
\li Lists describe relations between data, not appearance.
\ei

\end{frame}

\sect{Ordered list}
\bi
\li Only difference is the outer tag.
\li Don't have to renumber items.
\ei
\end{frame}

\sect{Nested lists}
\bi
\li Expressing more complex, hierarchical data.
\li Indentation can really help keep track of relationships.
\ei
\end{frame}

\sect{Indenting source code}
\bi
\li Indent each nested element.
\li Line up your elements.
\li Use spaces, not tabs.
\bi\li Configure your editor to use spaces for tabs.\ei
\li Use two spaces.
\li Closing tags should line up with opening tags.
\ei
\end{frame}


\sect{Building nested lists}
\bi
\li Create outer list first.
\li Add list items to outer list.
\li Validate before adding next list level.
\li Add first inner list.
\li Repeat until finished.
\li Validate frequently.
\ei
\end{frame}

\sect{Definition list}
\bi
\li Uses names instead of bullets or numbers.
\li Uses {\em definition terms} \verb|<dt>| and {\em definition
  descriptions} \verb|<dd>|. 
\ei
\end{frame}

\sect{Tables}
\begin{Verbatim}
  <table border="1">
  <tr>
  <th>  </th>
  <th>  </th>
  <th>  </th>
  </tr>
  <tr>
  <td> </td>
  <td> </td>
  <td> </td>
  </tr>
  </tr>
  <tr>
  <td> </td>
  <td> </td>
  <td> </td>
  </tr>
  </table>
\end{Verbatim}

\end{frame}

\sect{Building a table}
\bi
\li Plan ahead.
\li Create the headings.
\li Build a sample empty row.
\li Copy and paste the empty row to make as many  as you need.
\li Save, view, and validate.
\li Populate the table with the data.
\li Test and validate again.
\ei
\end{frame}



\end{document}
