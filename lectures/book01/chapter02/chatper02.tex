\documentclass{beamer}
\usepackage{fancyvrb}
\usepackage{hyperref}
\usepackage{alltt}
\usepackage{graphicx}

\newtheorem{theo}{Theorem}[section]


\newcommand{\myfig}[1]{\centerline{\includegraphics[scale=0.25]{figures/#1.png}}}
\newcommand{\myfigt}[1]{\centerline{\includegraphics[scale=0.2]{figures/#1.png}}}

\newcommand{\trans}[5]{
\begin{tabular}{|c|c|c|c|c|}\hline
#1 & #2 & #3 & #4 & #5 \\\hline
\end{tabular}
}

\newcommand{\arr}{&\rightarrow&}
\newcommand{\darr}{&\Rightarrow&}
\newcommand{\ar}{\ensuremath{\rightarrow}}
\newcommand{\dar}{\ensuremath{\Rightarrow}}
\newcommand{\bee}{\begin{eqnarray*}}
\newcommand{\eee}{\end{eqnarray*}}
\newcommand{\emptystring}{\ensuremath{\epsilon}}

\newcommand{\bi}{\begin{itemize}}
\newcommand{\li}{\item}
\newcommand{\ei}{\end{itemize}}

\newcommand{\sect}[1]{
\section{#1}
\begin{frame}[fragile]\frametitle{#1}
}
\newcommand{\sectc}[1]{
\section{#1}
\begin{frame}[fragile]\frametitle{#1}
\begin{columns}
}
\newcommand{\nc}[1]{\column{#1\textwidth}}
\newcommand{\ec}{\end{columns}}

\mode<presentation>
{
%  \usetheme{Madrid}
  % or ...

%  \setbeamercovered{transparent}
  % or whatever (possibly just delete it)
}

\usepackage[english]{babel}

\usepackage[latin1]{inputenc}

\title[HTML5 and CSS3, Chapter 2]
{
HTML5 and CSS3, Chapter 2
}

\subtitle{} % (optional)

\author[Geoffrey Matthews]
{Geoffrey Matthews}
% - Use the \inst{?} command only if the authors have different
%   affiliation.

\institute[WWU/CS]
{
  Department of Computer Science\\
  Western Washington University
}
% - Use the \inst command only if there are several affiliations.
% - Keep it simple, no one is interested in your street address.

\date{\today}

% If you have a file called "university-logo-filename.xxx", where xxx
% is a graphic format that can be processed by latex or pdflatex,
% resp., then you can add a logo as follows:

%\pgfdeclareimage[height=0.5cm]{university-logo}{WWULogoProColor}
%\logo{\pgfuseimage{university-logo}}

% If you wish to uncover everything in a step-wise fashion, uncomment
% the following command: 

%\beamerdefaultoverlayspecification{<+->}

\begin{document}

\begin{frame}
  \titlepage
\end{frame}


\newcommand{\myref}[1]{\small\item\url{#1}}
\newcommand{\myreft}[1]{\footnotesize\item\url{#1}}

%\begin{frame}
%  \frametitle{Outline}
%  \tableofcontents
%  % You might wish to add the option [pausesections]
%\end{frame}

\sect{HTML inherited many problems}

\bi
\li Browser manufacturers added exclusive features.
\li The distinction between meaning and layout was blurred.
\li Tables were used as a hack.
\li People started using tools to write pages.
\li The nature of the web was changing.
\li XHTML tried to fix things.
\li XHTML was difficult and complicated.
\ei

\end{frame}

\sect{XHTML had some good ideas}

\bi
\li All tags have endings.
\li Tags cannot be overlapped.
\bi
\li Bad: \verb|<a><b>my stuff</a></b>|
\ei
\li Everything's lowercase.
\li Attributes must be in quotes.
\li Layout must be separate from markup.
\bi
\li Bad: \verb|<font>|
\li Bad: \verb|<center>|
\ei
\li HTML5 is slightly less strict, but preserves these rules.
\ei
\end{frame}

\sect{Validation}

\bi
\li \url{https://validator.w3.org/}
\ei
\bi
\li Syntax verification for HTML
\li Can validate three ways:
\bi
\li by URL
\li by file upload
\li by direct input
\ei
\li Ox and Wheels example
\ei

\end{frame}

\sect{Tidy}
\bi
\li \url{http://infohound.net/tidy/}
\li Tidy doesn't just find the errors, it tries to fix them!
\li Tidy outputs XHTML by default.
\bi \li Deselect that box.\ei
\li If Tidy gets confused, it may just remove tags.
\li Set {\bf indent} to {\bf on}, so you can see if tags match up.
\ei
\end{frame}



\end{document}
