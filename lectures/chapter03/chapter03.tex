\documentclass{beamer}
\usepackage{fancyvrb}
\usepackage{hyperref}
\usepackage{alltt}
\usepackage{graphicx}

\newtheorem{theo}{Theorem}[section]


\newcommand{\myfig}[1]{\centerline{\includegraphics[scale=0.25]{figures/#1.png}}}
\newcommand{\myfigt}[1]{\centerline{\includegraphics[scale=0.2]{figures/#1.png}}}

\newcommand{\trans}[5]{
\begin{tabular}{|c|c|c|c|c|}\hline
#1 & #2 & #3 & #4 & #5 \\\hline
\end{tabular}
}

\newcommand{\arr}{&\rightarrow&}
\newcommand{\darr}{&\Rightarrow&}
\newcommand{\ar}{\ensuremath{\rightarrow}}
\newcommand{\dar}{\ensuremath{\Rightarrow}}
\newcommand{\bee}{\begin{eqnarray*}}
\newcommand{\eee}{\end{eqnarray*}}
\newcommand{\emptystring}{\ensuremath{\epsilon}}

\newcommand{\bi}{\begin{itemize}}
\newcommand{\li}{\item}
\newcommand{\ei}{\end{itemize}}

\newcommand{\sect}[1]{
\section{#1}
\begin{frame}[fragile]\frametitle{#1}
}
\newcommand{\sectc}[1]{
\section{#1}
\begin{frame}[fragile]\frametitle{#1}
\begin{columns}
}
\newcommand{\nc}[1]{\column{#1\textwidth}}
\newcommand{\ec}{\end{columns}}

\mode<presentation>
{
%  \usetheme{Madrid}
  % or ...

%  \setbeamercovered{transparent}
  % or whatever (possibly just delete it)
}

\usepackage[english]{babel}

\usepackage[latin1]{inputenc}

\title[HTML5 and CSS3, Chapter 3]
{
HTML5 and CSS3, Chapter 3
}

\subtitle{} % (optional)

\author[Geoffrey Matthews]
{Geoffrey Matthews}
% - Use the \inst{?} command only if the authors have different
%   affiliation.

\institute[WWU/CS]
{
  Department of Computer Science\\
  Western Washington University
}
% - Use the \inst command only if there are several affiliations.
% - Keep it simple, no one is interested in your street address.

\date{\today}

% If you have a file called "university-logo-filename.xxx", where xxx
% is a graphic format that can be processed by latex or pdflatex,
% resp., then you can add a logo as follows:

%\pgfdeclareimage[height=0.5cm]{university-logo}{WWULogoProColor}
%\logo{\pgfuseimage{university-logo}}

% If you wish to uncover everything in a step-wise fashion, uncomment
% the following command: 

%\beamerdefaultoverlayspecification{<+->}

\begin{document}

\begin{frame}
  \titlepage
\end{frame}


\newcommand{\myref}[1]{\small\item\url{#1}}
\newcommand{\myreft}[1]{\footnotesize\item\url{#1}}

%\begin{frame}
%  \frametitle{Outline}
%  \tableofcontents
%  % You might wish to add the option [pausesections]
%\end{frame}

\sect{Dreamweaver ({\em etc.)}: Pros}
\bi
\li WYSIWYG editing.
\li Templates
\li Site management
\ei

\end{frame}

\sect{Dreamweaver ({\em etc.)}: Cons}
\bi
\li Code maintenance
\li Vendor lock-in
\li Cost
\li Complexity
\bi\li \$400 and up, depending on packages \ei
\li Code understandability
\li Spotty standards compliance
\li Display variations
\bi \li WISIWYG is a lie \ei
\li Incompatability with other tools
\bi\li Content management systems \ei
\ei
\end{frame}

\sect{Online site builders}
\bi
\li Some CMSs allow you to build and modify a web page right in the
browser.
\li Further discussion in Book VIII of your text.
\li You should still learn HTML, CSS, {\em etc.}
\ei
\end{frame}

\sect{Roll your own webpage tools}
\bi
\li Enhanced text editors.
\li Browsers and plugins.
\li Server software
\bi\li Apache, PHP, MySQL, ...
\ei
\li Multimedia tools
\bi
\li images, videos, sounds, music, ...
\ei
\ei
\end{frame}

\sect{Text editors to avoid}
\bi
\li Microsoft Word
\li Windows Notepad
\li Mac TextEdit
\ei
\end{frame}

\sect{Suggested editors}
\bi
\li\url{https://notepad-plus-plus.org/}
\li\url{https://wiki.gnome.org/Apps/Gedit}
\li\url{http://www.vim.org/}
\li\url{https://www.gnu.org/software/emacs/}
\li\url{http://www.activestate.com/komodo-edit}
\ei
\end{frame}


\sect{A History of Web Browsers}
\bi
\li Mosaic/Netscape: the killer application
\bi\li 97\% market share at peak\ei
\li Microsoft Internet Explorer (IE)
\bi
\li First Browser War
\li Each browser added incompatible features
\li Microsoft eventually wins
\li IE becomes {\em the} standard
\ei
\li Firefox shakes up the world
\bi
\li Compliance to W3C standards
\li Tabbed browsing
\li Easy customization
\li Improved security
\ei
\li WebKit
\bi
\li Rendering engine used by both Safari and Chrome
\ei
\ei
\end{frame}

\sect{Microsoft Internet Explorer}
\bi
\li IE 10 has respectable HTML5 support.
\li IE 6 was the standard for many years, but largely incompatible
with new standards.
\ei
\end{frame}

\sect{Mozilla Firefox}
Improvement on IE from a programmer's point of view:
\bi
\li Better code view
\li Better error-handling
\li Great extensions
\li Multi-platform support
\ei
\end{frame}

\sect{WebKit/Safari}
\bi
\li Default for Mac, iPhone/iPad
\li Built on WebKit, same as Chrome
\ei
\end{frame}

\sect{Google Chrome}
\bi
\li Very fast, especially javascript
\li Complies well with new standards
\li Good developer toolkits available
\li Developer features:
\bi
\li Real-time page editing
\li Page outline
\li Realtime CSS edit
\li Network tab
\bi\li how long does each piece of a page take to load?\ei
\li Sources view
\li Console
\ei
\ei
\end{frame}

\sect{Opera}
\bi
\li Very good HTML5 compliance
\li Many game consoles and mobile devices use Opera
\ei
\end{frame}

\sect{Text-only browsers}
\bi
\li Very fast.
\li Auditory browsers read the contents of web pages.
\li Useful to understand how pages are used by people with visual
disabilities.
\li Useful to understnad how pages are viewed by search engines.
\ei
\end{frame}




\end{document}
